\documentclass[12pt,oneside,reqno]{amsart}
\usepackage[left=3cm,right=3cm,top=5cm,bottom=2cm]{geometry} % page settings
\usepackage{amsmath} % provides many mathematical environments & tools
\usepackage{amssymb}
\usepackage{tikz}

\begin{document}
\setlength{\parindent}{6pt}
\def\code#1{\texttt{#1}} %Code looking

\title{TIN093 Exercise week 3}
\author{Josefin Ondrus}
\date{\today}
\maketitle

%***********OK**************A
\textbf{A. Let $n$ and $m$ denote the number of nodes...}\\\\
%*************************
Given: $n,m$ - the number of nodes and edges in a connected graph.\\
Could $O(m)$ and $O(n+m)$ represent the time bound for the same algorithm?\\\\
I'll argument that they can. If we only look at the numbers and what type of graph it is (which we have to, since we have no further info about the algorithm), we could make some statements regarding the relationship between $n$ and $m$.\\\\
In a connected graph, the smallest number of edges possible is $n-1$.\\
Assuming the graph is undirected, the maximum number of edges possible is $\frac{n(n-1)}{2}$\\
Assuming the graph is directed, the maximum number of edges possible is $n(n-1)$.\\
(However, it will not matter whether the graph is directed or undirected so we assume it is undirected for now.)\\\\
According to the statements above. We know that $n-1 \leq m \leq \frac{n(n-1)}{2}$. This means that the only time $m$ could be bigger than $n$ is when $m$ is as small as possible. If we would compare the two time bounds in both cases we'll get the following differences:\\\\
For the smallest $m$:\\
$O(m)=O(n-1)$\\
$O(n+m)=O(n+(n-1))=O(2n-1)$\\
Taking only the dominating factor in consideration, we see that they are equal.\\\\
For the largest $m$:\\
$O(m)=O(\frac{n(n-1)}{2})=O(\frac{1}{2}(n^2-n))$\\
$O(n+m)=O(n+\frac{n(n-1)}{2})=O(\frac{1}{2}(2n+n^2-n))=O(\frac{1}{2}(2n^2+n))$\\ 
As with the smallest $m$ the dominating factor in both cases are equal. (If it would've been a directed graph, it would result in $O(2n^2+n)$), making no difference to th dominating factor.)\\\\
Hence we can argue that both time bounds can represent the same algorithm and none of them have to be mistaken. It is most likely a difference in how strict they've calculated the bounds. 


%***********OK**************B
\textbf{B. Why can we write $O(m log n)$ instead of...}\\\\
%*************************
As mentioned above, we can describe $m$ within an interval depending on $n$ as: $n-1 \leq m \leq \frac{n(n-1)}{2}$. If we would compare the two different expressions we can come to the following conclusions:\\\\
$O(mlog(n-1))$ is such a small difference from $O(mlogn)$ and they both have the same growth as $m,n$ increases.\\\\
$O(mlog(\frac{n(n-1)}{2}))=O(m(log(n(n-1))-log2))=O(m(logn+log(n-1)-log2))$, if we ignore the slower growing factors we get a time bound of $O(mlogn)$. \\\\
If it would have been a directed graph, the difference in the max number of edges would only affect the slower growing factors, resulting in the same result as for the undirected graph.\\\\

%************************C
\textbf{C. Suppose that we have executed depth-first search...}\\\\
%*************************
"Every cycle in the graph only consists of tree edges and one back edge." is ture.\\\\
Let's start with the definition of a back edge - "If in edge $(u,v)$, $v$ is the discovered already and $v$ is an ancestor, then $(u,v)$ is a back edge." This would clearly be the case in every cycle, since we always need to "close the cycle" for it to be a cycle and since we have an undirected graph, we cannot get cross-edges nor forward-edges in our DFS tree, hence, the cycle can only consist of tree edges and one back edge.\\\\
The algorithm I propose is a DFS where we for every time we check a node and it is marked as visited, we check if it is an ancestor of our current node. If so, we return true, as the graph then have a cycle. Since the time for a DFS is $O(m+n)$ and our extra check gives us a constant addition to the time complexity, we can ignore this and hence, our algorithm has a time bound of$O(m+n)$.

%*************************D
\textbf{D. Can an articulation point belong to...}\\\\
%*************************
Yes! See below example where node 4 is an articulation point and a part of the circle $4->5->2->3->4$.\\\\
\begin{tikzpicture}
  [scale=.5,auto=right,every node/.style={circle,fill=white!20}]
  \node (n1) at (1,6) {1};
  \node (n6) at (1,10) {6};
  \node (n4) at (4,8)  {4};
  \node (n5) at (8,9)  {5};
  \node (n2) at (9,6)  {2};
  \node (n3) at (5,5)  {3};

  \foreach \from/\to in {n6/n4,n4/n5,n6/n1,n2/n5,n2/n3,n3/n4}
    \draw (\from) -- (\to);

\end{tikzpicture}\\\\
%*************************E
\textbf{E. We have seen how BFS can be used to test...}\\\\
%*************************

%*************************F
\textbf{F. We run Kruskal’s algorithm to get a clustering...}\\\\
%*************************

%*************************EXAM
\textbf{Can be exam-related. Suppose that a cleaning robot can walk...}\\\\
%*************************



\end{document}