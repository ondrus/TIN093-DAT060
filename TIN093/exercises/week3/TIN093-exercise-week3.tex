\documentclass[12pt,oneside,reqno]{amsart}
\usepackage[left=3cm,right=3cm,top=5cm,bottom=2cm]{geometry} % page settings
\usepackage{amsmath} % provides many mathematical environments & tools
\usepackage{amssymb}

\begin{document}
\setlength{\parindent}{6pt}
\def\code#1{\texttt{#1}} %Code looking

\title{TIN093 Exercise week 3}
\author{Josefin Ondrus}
\date{\today}
\maketitle

%*************************Q1
\textbf{A. Explain in your words why the dynamic programming algorithm...}\\\\
%*************************
The dynamic programming programming algorithm for Knapsack is nor a polynomal-time algorithm since 

%*************************Q2
\textbf{B. Explain the proposed dynamic programming formula...}\\\\
%*************************
The proposed dynamic programming formula is

We get this formula through

It is correct since

We can compute the OPT values in the following order

%************************Q3
\textbf{C. Now something more ambitious and tricky:...}\\\\
Given:

Goal:

The naive algorithm is $O(n^6)$ since

We can reduce the exponent by using dynamic programming. HOW? -you ask..

Compute
%*************************

%*************************Q4
\textbf{Can be exam-related. We modify the Weighted Interval Scheduling...}\\\\
%*************************
Given:

Goal:

Compute 

If we analyze the time complexity of the computation above

\textbf{Can be exam-related. Extra question: Is Fixed Cardinality Weighted..}\\\\
%*************************PROGRAMMING
\textbf{Recommended Programming. You may find it easier to get...}\\\\
%********************

\end{document}