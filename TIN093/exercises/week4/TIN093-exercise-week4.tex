\documentclass[12pt,oneside,reqno]{amsart}
\usepackage[left=3cm,right=3cm,top=5cm,bottom=2cm]{geometry} % page settings
\usepackage{amsmath} % provides many mathematical environments & tools
\usepackage{amssymb}

\begin{document}
\setlength{\parindent}{6pt}
\def\code#1{\texttt{#1}} %Code looking

\title{TIN093 Exercise week 4}
\author{Josefin Ondrus}
\date{\today}
\maketitle

%*************************A
\textbf{A. An (idealized, fictive) country is divided into...}\\\\
%*************************
Given: $r$ - the number of regions the country and the regions are divided in.\\
$d$ - the level of regions.\\
$c(i)$ the cost of every region on level $i$.\\
Goal: Calculate total cost.\\\\
What we want is the total sum $S$ of the country. If we visualize the structure as a symmetric tree with $d$ levels. Then we know that the number of regions in every level $i$ is $r^{i-1}$, which gives us the cost of every level to $c(i)*r^{i-1}$. If we now calculate the total sum of all the levels, we get $\sum_{i=1}^{d}c(i)*r^{i-1}$, which should give us the total cost of running the country. Tada!\\\\

%***********OK**************C
\textbf{C. How much time is needed to decide whether...}\\\\
%*************************
Assuming the sets are in two different arrays $A_1$ and $A_2$:\\\\
If we don't sort the arrays, in worst case, we need to check the first number in $A_1$ to all numbers in $A_2$, the second number in $A_1$ to $n-1$ numbers in $A_2$ giving us a calculation time of $O(n^2-\frac{n(n+1)}{2})$ i.e. time complexity of $O(n^2)$.\\\\
However, if we sort both arrays (in $2*nlog(n)$ time) and then compare every number in $A_{1i}$ to the corresponding number $A_{2i}$ worst case scenario is if only two numbers are different or if the arrays are equal. We then get a calculation time of $O(n)$. There for, in total $O(2*nlog(n)) + O(n)$ giving us a time complexity of $O(nlog(n))$. WOHO! Better than $O(n^2)!$\\\\
\newpage
%************OK*************EXAM
\textbf{Can be exam-related. A certain industrial product will be broken...}\\\\
%*************************
Given: $n$ different values of $f$.\\
Goal: Find the discrete value $f$ for which if the force is $>f$, the product breaks.\\\\
Prove that $O(n)$ tests are needed in the worst case if we can only allow 1 bad test:\\
Assuming the values are sorted in a way that $f_1 < f_2 < ... < f_n$, we encounter worst case scenario when $f=f_n$. This is because we need to go through every single value in increasing order to ensure us that we do not miss the correct $f$. (i.e. starting at $f_1, f_2, ..., f_n$). If we try to reduce the number of tests we would obviously have to skip some tests. This could reduce the number of possible values of $f$ but we would never know for sure the exact value of $f$ since we've used our only bad case and more possible values are still to be tested.\\\\
Example: Lets say that we'd skip every other value of $f$. This would reduce the number of tests needed by half (i.e. $O(\frac{n}{2})$), but we wouldn't know if $f=f_i$ of if $f=f_{i-1}$ (and $f=f_n$ or $f=f_{n-1}$ in worst case).\\\\
Furthermore: Can you save tests if some more bad tests are permitted?\\
Yes! If we were give just two bad tests in total, we could reduce the number of tests. For example by half, using the strategy mentioned above.\\

Assuming the values are sorted increasing in an array, we could also reduce the number of tests by calculating min($\frac{n}{i}+i-1$), where $n$ still is the number of values and $i$ is a variable representing the $f_i$ we start our search on. If we get a fail on $f_i$, we start from $f_1$ and test out self up to at most $f_i-1$. If the we don't get a fail on $f_i$, we test $f_2*i$. If the product doesn't break at $f_2*i$, we try $f_3*i$, and so on. We get worst case if we don't get a bad until we try $f_n$, forcing us to test $i$ times and then another $\frac{n}{i}-1$ times to find the correct $f$.\\\\
If we are allowed to have log(n) bad test, we could determine $f$ in $O(log(n))$ tests by binary search.


\end{document}