\documentclass[12pt,oneside,reqno]{amsart}
\usepackage[left=3cm,right=3cm,top=5cm,bottom=2cm]{geometry} % page settings
\usepackage{amsmath} % provides many mathematical environments & tools
\usepackage{amssymb}

\begin{document}
\setlength{\parindent}{6pt}
\def\code#1{\texttt{#1}} %Code looking

\title{TIN093 Exercise week 3}
\author{Josefin Ondrus}
\date{\today}
\maketitle



%*************************Q4
\textbf{Can be exam-related. A certain industrial product will be broken...}\\\\
%*************************
Given: \\
$P$ the set of $n$ intervals $p$ such that $P={p_1, p_2,...p_n}$.\\
Every interval $p_i, i=1,...,n$ has a corresponding value $v_i$ and a start and finish $[s_i,f_i]$ and $f_i < s_j$.\\
$k \in \mathbb{N}$ the number of intervals we want to have in our set.\\

Goal: To select a subset of $P$ of pairwise disjoint intervals such that $f_i < s_j$. The number of selected intervals needs to be equal to $k$, ($\sum\limits_{i\in P}p_i=k$) in such a way that we maximize the sum of the value $v = \sum\limits_{i\in P}v_i$.

Dynamic programming: This problem clearly have some equalities with the Knapsack problem. We could see it as we want to pack exactly $k$ pairwise disjoint items $p$ at the same time as we maximize the value $v$.

Did not have time to write down more than this but would like some comments on the formulation. :)

\end{document}