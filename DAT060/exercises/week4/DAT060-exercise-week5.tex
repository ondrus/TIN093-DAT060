\documentclass[12pt,oneside,reqno]{amsart}
\usepackage[left=3cm,right=3cm,top=3cm,bottom=2cm]{geometry} % page settings
\usepackage{amsmath} % provides many mathematical environments & tools
\usepackage{amssymb}
\usepackage{natded}
\usepackage{booktabs}

\begin{document}
\setlength{\parindent}{6pt}
\def\code#1{\texttt{#1}} %Code looking
\def\ra{\rightarrow{}} %Short for right arrow
%\def\proof{\proofline{\amsmath}{}} %Short for right arrow
\newcommand{\itab}[1]{\hspace{0em}\rlap{#1}}
\newcommand{\tab}[1]{\hspace{.2\textwidth}\rlap{#1}}
\newcommand\deff{\mathrel{\stackrel{\makebox[0pt]{\mbox{\normalfont\tiny def}}}{=}}}
\raggedbottom

\title{Josefin Ondrus\\ondrus@student.chalmers.se}
\author{DAT060 Exercise week 5}
\date{\today}
\maketitle

%************DOUBLE CHECK************Q1
\textbf{Problem 1}\\
%*************************
\textbf{Input: A formula $\varphi$ of predicate logic.}\\
\textbf{Question: Is $\varphi$ satisfiable?}\\\\
We can only prove that $\varphi$ is satisfiable by proving $M \models_l \varphi$.
This is however undecidable since no program exists which, given any $\varphi$, decides whether $\models \varphi$.\\\\

%***********OK*************Q2
\textbf{Problem 2}\\
%*************************
\textbf{$\varphi := \forall xP(x,f(x)) \land \forall x\exists y \neg P(x,y)$}\\

\textbf{$M \models \varphi$}\\
$A^M \deff \{m,n \in \mathbb{N}\}$\\
$f(m) \deff m$\\
$P(m,n) \deff m=n$\\\\
\textbf{$M' \not\models \varphi$}\\
$A^M \deff \{m,n \in \mathbb{Z}\}$\\
$f(m) \deff -m$\\
$P(m,n) \deff n=-m$\\\\

%**********OK**************Q2
\textbf{Problem 3}\\
%*************************

\textbf{a) }$\forall x (P(x) \ra \neg Q(x)) \vdash \neg \exists x (P(x) \land Q(x))$
	\[
	\Jproof{
		\proofline{\forall x (P(x) \ra \neg Q(x))}{prem}
		\cablk{
			\proofline{\exists x (P(x) \land Q(x))}{ass}
			\cablk{
				\proofline{x_0\ P(x_0) \land Q(x_0)}{}
				\proofline{P(x_0) \ra \neg Q(x_0)}{$\forall x e\ 1$}
				\proofline{P(x_0)}{$\land e_1\ 3$}
				\proofline{Q(x_0)}{$\land e_2\ 3$}
				\proofline{\neg Q(x_0)}{$\ra e\ 4,5$}
				\proofline{\perp}{$\neg e\ 6,7$}
			}
			\proofline{\perp}{$\exists x e\ 2,3-8$}
		}
		\proofline{\neg \exists x(P(x) \land Q(x))}{$\neg i\ 2-9$}\\
	}
	\]

\textbf{b) }$\forall x (\neg Q(x) \lor S), \exists y(P(y) \ra Q(f(y))) \vdash \exists z (P(z) \ra S)$
	\[
	\Jproof{
		\proofline{\forall x (\neg Q(x) \lor S)}{prem}
		\proofline{\exists y(P(y) \ra Q(f(y)))}{prem}
		\cablk{
			\proofline{x_0\ P(x_0 \ra Q(F(x_0)))}{}
			\proofline{\neg Q(f(x_0)) \lor S}{$\forall xe\ 1$}
			\cablk{
				\proofline{P(x_0)}{ass}
				\cablk{
					\proofline{\neg Q(f(x_0))}{ass}
					\proofline{Q(f(x_0))}{$\ra e\ 5,3$}
					\proofline{\perp}{$\neg e\ 6,7$}
					\proofline{S}{$\perp e\ 8$}
				}
				\cablk{
				\proofline{S}{ass}
				}
			\proofline{S}{$\lor e\ 4,6-9,10$}	
			}
		\proofline{P(x_0)\ra S}{$\ra i\ 5-11$}	
		}
	\proofline{P(z) \ra S}{$\exists ye\ 2,3-12$}\\
	}
	\]
%************************Q4
\textbf{Problem 4}\\
%*************************

\textbf{a) }$\neg \forall x P(x) \vdash \forall x \neg P(x)$\\\\
Given the counter model:\\
$P^M \deff \mathbb{N}$\\
$P(m) \deff m=3$\\
In text: Not all natural numbers are equal to three, but all natural numbers is not separated from three.\\\\
\textbf{b) }$\exists x P(x), \exists y Q(y) \vdash \exists z(P(x) \land Q(z))$
Given the counter model:\\
$P^M \deff$ all animals\\
$P(m) \deff m$ is a dog\\
$Q(m) \deff m$ is a cat\\
In text: There exist animals that are dogs, and there exist animals that are cats, but there is no animal that is both cat and dog at the same time.\\\\
%************************Q2
\textbf{Problem 5}\\
%*************************
$\exists x \exists y (x\not= y \land \forall z (z=x \lor z=y))$ Where $x\not=y$ is $\neg(x=y)$\\\\
The formula can only be true for the models where the following holds:\\
$A \deff \{m,n | m\not=n \}$.\\\\
We need to constrain the universe to two elements since it must hold for all $z$. As soon as we have three or more (or less than two) elements in the universe we know that the expression $(z=x \lor z=y)$ can't be true for all $z$ since $x$ and $y$ only can cover exactly two of the values, and $z$ needs to cover all values.
\end{document}