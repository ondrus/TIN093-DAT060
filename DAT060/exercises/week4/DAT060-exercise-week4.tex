\documentclass[12pt,oneside,reqno]{amsart}
\usepackage[left=3cm,right=3cm,top=3cm,bottom=2cm]{geometry} % page settings
\usepackage{amsmath} % provides many mathematical environments & tools
\usepackage{amssymb}
\usepackage{natded}
\usepackage{booktabs}

\begin{document}
\setlength{\parindent}{6pt}
\def\code#1{\texttt{#1}} %Code looking
\def\ra{\rightarrow{}} %Short for right arrow
%\def\proof{\proofline{\amsmath}{}} %Short for right arrow
\newcommand{\itab}[1]{\hspace{0em}\rlap{#1}}
\newcommand{\tab}[1]{\hspace{.2\textwidth}\rlap{#1}}
\raggedbottom

\title{Josefin Ondrus\\ondrus@student.chalmers.se}
\author{DAT060 Exercise week 4}
\date{\today}
\maketitle

%************************Q1
\textbf{Problem 1}\\
%*************************
\textbf{a)}$\forall x f(f(x))=f(x), f(b)=c \vdash c=f(c)$
	\[
	\Jproof{
		\proofline{\forall x f(f(x))=f(x)}{prem}
		\proofline{f(b)=c}{prem}
		\proofline{f(f(b))=f(b)}{$\forall x e\ 1$}
		\proofline{f(c)=c}{$=e\ 2,3$}
		\proofline{c=c}{$=i$}
		\proofline{c=f(c)}{$=e\ 4,5$}\\\\
	}
	\]

\textbf{b)}$\forall x \forall y (x=g(y) \ra f(x)=y) \vdash \forall x f(g(x))=x$
	\[
	\Jproof{
		\proofline{\forall x \forall y (x=g(y) \ra f(x)=y)}{prem}
		\cablk{
			\proofline{x_0}{}
			\proofline{x_0=g(x_0)}{ass}
			\proofline{\forall y(x_0=g(y) \ra f(x_0)=y)}{$\forall xe\ 1$}
			\proofline{x_0=g(x_0) \ra f(x_0)=x_0}{$\forall ye\ 4$}
			\proofline{f(x_0)=x_0}{$\ra e\ 3,5$}
			\proofline{f(g(x_0))=x_0}{$=e\ 3,6$}
		}
		\proofline{\forall x f(g(x))=x}{$\forall x i\ 2-7$}
	}
	\]

%************************Q2
\textbf{Problem 2}\\
%*************************
$\varphi := \forall x \exists y \exists z (P(x,y) \land P(y,z) \land \forall w (P(w,x) \ra P(w,z)))$\\\\
\textbf{a)} $P^{M_0} := \{(m,n) | m<n $ and $m,n \in \mathbb{N}\}$\\\\
$(1)$ If $x<y$ and  $y<z$ we know that $x<z$ since $"<"$ is a transitive operator.\\
$(2)$ If $w<x$ and we know that $x<z$ from $(1)$, we know that $w<z$ for all $w$.\\
And we get that $P^{M_0}$ satisfies $\varphi$ from $(1)$ and $(2)$.\\\\

\textbf{b)} $P^{M_1} := \{(m,2*m) | m \in \mathbb{N}\}$\\\\
Lets define $m$ as $m_1=x$, $m_2=y$, $m_3=z$ and $m_4=w$.\\ 
From $P^{M_1} := \{(m,2*m)$ we then get:\\
$(1)\ 2m_1=m_2$\\
$(2)\ 4m_1=2m_2=m_3$\\
and for all $m_4$: if $m_4=2m_1$ holds, $m_4=2m_3$ should hold ($\forall w (P(w,x) \ra P(w,z))$) but since $2m_3=8m_1$ (from $(2)$) and we know that $2m_1 \not= 8m_1$ for all $m_1 \in \mathbb{N}$ we also know that $P^{M_1}$ does not satisfy $\varphi$.\\\\

%************************Q2
\textbf{Problem 3}\\
%*************************
\textbf{a) $\forall x \exists y R(x,y) \vdash \exists y \forall x R(x,y)$}\\\\
Assume the model from problem $2\ b)$ where $P^{M_1} := \{(m,2*m) | m \in \mathbb{N}\}$\\\\The premise states that for all $m$ there exists a $n$ such that $n = 2*m$. i.e if we take an arbitrary $m \in \mathbb{N}$, we can always find a $n$ twice as big.\\\\
The conclusion states that there exists a $n$ such that $n = 2*m$ for all m. i.e there exists a $n$ that for any $m \in \mathbb{N}$ is twice as big. \\Thus, the conclusion is not valid since there exists no such $n\in \mathbb{N}$\\\\

\textbf{b) $\forall x(P(x)\lor Q(x)) \vdash \forall x P(x) \lor \forall x Q(x)$}\\\\
Given a model M where:\\
$P^M := \{(m)|m<0$ and $m\in\{-1,1\}\}$\\
$Q^M := \{(m)|m>0$ and $m\in\{-1,1\}\}$\\\\
We know that the premise is always valid for $M$ but never valid for the conclusion since not P nor Q is valid for all $x$, ever.

\textbf{c) $\forall x \exists y (P(x) \ra Q(y)) \vdash \forall x (P(x) \ra \exists y Q(y))$}\\
	\[
	\Jproof{
		\proofline{\forall x \exists y (P(x) \ra Q(y))}{prem}
		\proofline{\exists y (P(x_0) \ra Q(y))}{$\forall xe\ 1$}
		\cablk{
			\proofline{x_0}{\ }
			\cablk{
				\proofline{y_0\ P(x_0)\ra Q(y_0)}{ass}
				\cablk{
					\proofline{P(x_0)}{ass}
					\proofline{Q(y_0)}{$\ra e\ 4,5$}
					\proofline{\exists yQ(y)}{$\exists yi\ 6$}
				}
				\proofline{P(x_0)\ra \exists yQ(y)}{$\ra i\ 5-7$}
			}
			\proofline{P(x_0) \ra \exists y Q(y)}{$\exists ye\ 2, 4-8$}
		}
		\proofline{\forall x (P(x) \ra \exists y Q(y))}{$\forall i\ 3-9$}\\\\
	}
	\]

\textbf{d) $\forall x R(x,x), \forall x \forall y(R(x,y) \ra R(y,x)) \vdash \forall x \forall y \forall z(R(x,y) \land R(y,z) \ra R(x,z)) $}\\\\
For the model where $R^M := \{(a,a), (b,b), (c,c), (a,b), (b,a), (a,c), (c,a)\}$ The conclusion does not hold for there exists no $(b,c)$ and therefor the sequent is not valid.


\end{document}