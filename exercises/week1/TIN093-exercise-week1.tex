\documentclass[12pt,oneside,reqno]{amsart}
\usepackage[left=3cm,right=3cm,top=5cm,bottom=2cm]{geometry} % page settings
\usepackage{amsmath} % provides many mathematical environments & tools
\usepackage{amssymb}

\begin{document}
\setlength{\parindent}{6pt}
\def\code#1{\texttt{#1}} %Code looking

\title{TIN093 Exercise week 1}
\author{Josefin Ondrus}
\date{\today}
\maketitle

%***********OK?**************Q1
\textbf{What does $n^{O(1)}$ mean?}\\\\
%*************************
Since the exponent is constant we can write $n^{O(1)}=n^c$ which is polynomial time in $n$.\\\\

%**********OK***************Q2
\textbf{Is it correct to say $O(n+n+...+n) = O(n)$ or not? Explain}\\\\
%*************************
If we reformulate the expression we get $n+n+\dots+n=cn$ where $c \in \mathbb{N}$, $0 \leq c$. The purpose of $O(\cdot)$ is to define an upper bound based on the dominant factor as $n$ approaches infinity. Since $c$ is a constant value, $n$ is the dominant factor. The execution time based on $cn$ will be greater than $O(n)$ for a finite number of instances but will eventually not grow faster. \\\\

%***********OK**************Q3
\textbf{Suppose you have three algorithms (for whatever problem) that need $O(n^2)$, $O(2^n)$, and $O(n!)$ time, respectively. When the speed of your machine is doubled, how does this affect the size of inputs you can handle?}\\\\
%*************************
If we double the computation speed, the actual time it takes to run each one of the algorithms will halve (since we can compute the double amount of instructions in the same time). However, the time complexity will remain the same since we would express the calculation time as a constant factor: $\frac{1}{2}$. This will perhaps be noticeable for small values of $n$ but as $n$ increases, this constant's effect will decrease substantially. I would say that the speed of the machine does not affect the size of inputs I can handle.\\\\

%***********OK**************Q4
\textbf{Suppose you have an algorithm with two nested for-loops where the loop indices run through at most $n$ different values (such as: \code{for i = 1 to n do for j = i to n do [some elementary operation with $i$ and $j$] endfor endfor”)}, and another part of the algorithm is just one such for-loop with at most $n$ values. What would be the time complexity in O-notation? Explain in detail.}\\\\
%*************************
Lets call the double nested for-loop $(1)$ and the single for-loop $(2)$. Since the for-loops can run through at most $n$ different values, worst case scenario for $(1)$ will be reading $f(n)=n*n$ times, and worst case for $(2)$ is reading $g(n)=n$ times. This gives us a time complexity of $O(f(n)+g(n))$ (+ some constants representing the elementary operation). From the properties of O we know that\\\\ 
$O(f(n)+g(n))=O(max\{f(n), g(n)\})$\\\\
and for this algorithm: $g(n) \leq f(n)$ since $0 \leq n, n\in n^2$ \\\\
which gives us the time complexity $O(f(n))=O(n^2)$. Since the time for executing constants depends on the computer hardware we choose to disregard this.\\\\

%*************************Q5
\textbf{What is the worst-case time complexity of adding 1 to (or subtracting 1 from) an integer with $n$ digits, if we count operations with digits as elementary operations?}\\
%********************
$n$ ex: 999999999 or 10000000000

%*************************Q6
\textbf{What is the worst-case time complexity of comparing two integers $x, y$ with $n$ digits, if we count operations with digits as elementary operations? Comparing means to decide whether $x < y, x = y, or x < y$.}\\
%********************
$n$ ex: x=5555551 y=5555552

%*************************Q7
\textbf{How many digit operations do you need to compute the sum $1+2+3+\dots+n?$ Try to come up with a bound in O-notation which is as good as possible.}\\
%********************


%*************************Q8
\textbf{Give a rigorous proof that $m$ integers of $n$ digits can be added in $O(mn)$ time. (Intuitively this should be true, but the details are not so obvious.)}\\
%********************

%*************************Q9
\textbf{Let $p$ and $q$ be integers with $n$ and $m$ digits, respectively. In the following
suppose that both multiplication (calculating $pq$) and division with remainder
requires $O(nm)$ time. Division with remainder means to calculate the
integer part $\frac{p}{q}$ of the ratio, as well as the remainder $p$ mod $q$.
As is well known, an integer $p$ is a prime number if $p$ is divisible (with
remainder zero) only by 1 and $p$. We want to decide whether a given number
$p$ with $n$ digits is prime. A naive algorithm just closely follows the definition:
It divides $p$ by all integers from $2$ to $p−1$ and checks that the remainders
are nonzero. Derive a time bound in O-notation, both as a function of $n$
and as a function of $p$.}\\
%********************

\end{document}